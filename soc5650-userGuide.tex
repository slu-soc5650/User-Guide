\documentclass[]{book}
\usepackage{lmodern}
\usepackage{amssymb,amsmath}
\usepackage{ifxetex,ifluatex}
\usepackage{fixltx2e} % provides \textsubscript
\ifnum 0\ifxetex 1\fi\ifluatex 1\fi=0 % if pdftex
  \usepackage[T1]{fontenc}
  \usepackage[utf8]{inputenc}
\else % if luatex or xelatex
  \ifxetex
    \usepackage{mathspec}
  \else
    \usepackage{fontspec}
  \fi
  \defaultfontfeatures{Ligatures=TeX,Scale=MatchLowercase}
\fi
% use upquote if available, for straight quotes in verbatim environments
\IfFileExists{upquote.sty}{\usepackage{upquote}}{}
% use microtype if available
\IfFileExists{microtype.sty}{%
\usepackage{microtype}
\UseMicrotypeSet[protrusion]{basicmath} % disable protrusion for tt fonts
}{}
\usepackage[margin=1in]{geometry}
\usepackage{hyperref}
\hypersetup{unicode=true,
            pdftitle={SOC 4650/5650 Users Guide},
            pdfauthor={Christopher Prener, Ph.D.},
            pdfborder={0 0 0},
            breaklinks=true}
\urlstyle{same}  % don't use monospace font for urls
\usepackage{natbib}
\bibliographystyle{apalike}
\usepackage{longtable,booktabs}
\usepackage{graphicx,grffile}
\makeatletter
\def\maxwidth{\ifdim\Gin@nat@width>\linewidth\linewidth\else\Gin@nat@width\fi}
\def\maxheight{\ifdim\Gin@nat@height>\textheight\textheight\else\Gin@nat@height\fi}
\makeatother
% Scale images if necessary, so that they will not overflow the page
% margins by default, and it is still possible to overwrite the defaults
% using explicit options in \includegraphics[width, height, ...]{}
\setkeys{Gin}{width=\maxwidth,height=\maxheight,keepaspectratio}
\IfFileExists{parskip.sty}{%
\usepackage{parskip}
}{% else
\setlength{\parindent}{0pt}
\setlength{\parskip}{6pt plus 2pt minus 1pt}
}
\setlength{\emergencystretch}{3em}  % prevent overfull lines
\providecommand{\tightlist}{%
  \setlength{\itemsep}{0pt}\setlength{\parskip}{0pt}}
\setcounter{secnumdepth}{5}
% Redefines (sub)paragraphs to behave more like sections
\ifx\paragraph\undefined\else
\let\oldparagraph\paragraph
\renewcommand{\paragraph}[1]{\oldparagraph{#1}\mbox{}}
\fi
\ifx\subparagraph\undefined\else
\let\oldsubparagraph\subparagraph
\renewcommand{\subparagraph}[1]{\oldsubparagraph{#1}\mbox{}}
\fi

%%% Use protect on footnotes to avoid problems with footnotes in titles
\let\rmarkdownfootnote\footnote%
\def\footnote{\protect\rmarkdownfootnote}

%%% Change title format to be more compact
\usepackage{titling}

% Create subtitle command for use in maketitle
\newcommand{\subtitle}[1]{
  \posttitle{
    \begin{center}\large#1\end{center}
    }
}

\setlength{\droptitle}{-2em}
  \title{SOC 4650/5650 Users Guide}
  \pretitle{\vspace{\droptitle}\centering\huge}
  \posttitle{\par}
  \author{Christopher Prener, Ph.D.}
  \preauthor{\centering\large\emph}
  \postauthor{\par}
  \predate{\centering\large\emph}
  \postdate{\par}
  \date{2016-10-20}

\usepackage{booktabs}

\begin{document}
\maketitle

{
\setcounter{tocdepth}{1}
\tableofcontents
}
\chapter*{Preface}\label{preface}
\addcontentsline{toc}{chapter}{Preface}

This text is a companion document for \textbf{SOC 4650/5650 -
Introduction to Geographic Information Sciences}. It is designed to help
you be \emph{successful} in this course. The idea behind a course Users
Guide is to create a reference for many of the intangible, subtle or
disparate skills and ideas that contribute to being a successful
researcher. In creating a Users Guide, I draw inspiration from the work
of Donald Knuth.\footnote{\href{https://en.wikipedia.org/wiki/Donald_Knuth}{Donald
  Knuth} is the developer of
  \href{https://en.wikipedia.org/wiki/TeX}{TeX}, a computer typesetting
  system that is widely used today for scientific publishing in the form
  of \href{https://en.wikipedia.org/wiki/LaTeX}{LaTeX}. He also
  established the concept of
  \href{https://en.wikipedia.org/wiki/Literate_programming}{literatue
  programming}, which forms the basis of some of the practices we will
  follow with Stata this semester.} Knuth has discussed his experiences
in designing new software languages, nothing that the developer of a new
language

\begin{quote}
\ldots{}must not only be the implementer and the first large-scale user;
the designer should also write the first user manual\ldots{} If I had
not participated fully in all these activities, literally hundreds of
improvements would never have been made, because I would never have
thought of them or perceived why they were important\ldots{}
\end{quote}

While there is nothing particularly new about what I am writing here,
and I am certainly not developing a new language for computing, the goal
of the \textbf{Users Guide} remains similar to Knuth's experience.
However, by distilling some of key elements for making a successful
transition to being a \emph{professional developer} of knowledge rather
than a \emph{casual consumer}, I hope to both improve the course
experience itself and also create an environment that fosters a
successful learning experience for you.

If you read through the course objectives included in the syllabus, you
will note that creating maps is only one of them. As much as this is a
GIS course, it is a course in research methods. We are concerned not
just with any research methods, but \emph{high quality} research methods
and the \emph{process} of conducting research.

There is an important distinction to be made between research methods
and methodology.\footnote{Methodology refers to the \emph{reasons} for
  implementing a particular approach to research while method refers to
  the tools used in the research process
  (\href{https://books.google.com/books?id=Dq05AwAAQBAJ\&lpg=PA24\&dq=student's\%20guide\%20to\%20methodology\&pg=PP1\#v=onepage\&q\&f=false}{Clough
  and Nutbrown 2012}:25).} Research methods are the mental habits and
technical practices that make you a successful researcher. Some of the
skills and techniques that we will discuss this semester are not taught
as often in graduate programs. Instead, they are often the products of
``learning the hard way''. These ``habits of mind and habits of method''
are broadly applicable across methodologies and disciplines.

\chapter{Getting Started}\label{gettingStarted}

Before you begin the semester, there are a number of things that I
recommend that you do to help set yourself up for success. Before you do
\emph{anything} else, you should read through the
\href{}{\textbf{Syllabus}} and the \href{}{\textbf{Reading List}}. Make
sure you have a good sense of what is \emph{required} for the course. If
you have questions, bring them to the first day of class!

\section{Prep Your Computer}\label{prep-your-computer}

Before you do anything else for this course, make sure you get your
computer ready for the work you are about to undertake:

\begin{enumerate}
\def\labelenumi{\arabic{enumi}.}
\tightlist
\item
  Make sure your operating system is up-to-date. If you are able, I
  would also recommend upgrading your computer to the most recent
  release of its operating system that the computer can run.
\item
  We'll be sharing computer files throughout the semester, so you should
  ensure that you have functioning anti-virus software and that it is
  up-to-date.
\item
  You'll also need to download files, so you'll need to make sure you
  have some free space on your hard drive. If you have less than 10GB of
  free space, you should de-clutter!
\item
  Make sure you know how to access your computer's file management
  system.

  \begin{itemize}
  \tightlist
  \item
    On macOS, this means being comfortable with Finder.app.
  \item
    On Windows, this means being comfortable with Windows Explorer.
  \end{itemize}
\end{enumerate}

This of course assumes that you own a computer. Owning a computer is not
required for this course. All students who are enrolled in SOC 4650 or
SOC 5650 will be given 24-hour swipe access (\emph{just what you always
wanted!}) to Morrissey Hall to facilitate access to lab computers.

\section{Create Accounts}\label{create-accounts}

There are two major web services that we will use this semester, and
you'll need to create accounts for both:

\begin{itemize}
\tightlist
\item
  \textbf{GitHub} - you can sign-up at
  \href{https://github.com}{GitHub.com}. Once you've signed up, fill out
  your profile, set-up
  \href{https://help.github.com/articles/about-two-factor-authentication/}{two-factor
  authentication}, and let Chris know (via
  \href{mailto:prenercg@slu.edu}{email}) what your user name is. Once he
  has it, he can add you to the
  \href{https://github.com/slu-soc5650}{SOC 4650/5650} organization.
\item
  \textbf{Slack} - you can ask Chris (via
  \href{mailto:prenercg@slu.edu}{email}) for an invitation to sign-up
  for our team. Once the sign-up process is complete, you can log-in by
  going to our team's \href{https://slu-soc5650.slack.com}{Slack site}.
  Fill out your profile, set-up
  \href{https://get.slack.help/hc/en-us/articles/204509068-Set-up-two-factor-authentication}{two-factor
  authentication}, and change your timezone.
\end{itemize}

\section{Download and Install
Software}\label{download-and-install-software}

There are a number of software applications that we will use this
semester. Most of them are free, and I recommend downloading those free
ones right away. All of these applications are available for macOS and
Windows.

\begin{itemize}
\tightlist
\item
  \textbf{Atom} - Atom is a flexible, open-source text editor that is
  produced by GitHub. You can download it from Atom's
  \href{https://atom.io}{website}.
\item
  \textbf{GitHub Desktop} - GitHub makes a desktop client that you can
  use to easily interact with repositories that are stored on the site.
  You can download it from GitHub's
  \href{https://desktop.github.com}{website} after you sign-up for an
  account there. You'll need that account information to complete the
  desktop client's set-up process.
\item
  \textbf{Slack} - Slack has a number of applications for desktop and
  mobile operating systems. I recommend downloading Slack on your
  personal computer, and optionally installing it on your mobile device
  as well. You can download their desktop applications from their
  \href{https://slack.com/downloads}{website} and the mobile
  applications from your App Store.
\end{itemize}

\subsection*{\texorpdfstring{For Graduate Students
\emph{only}}{For Graduate Students only}}\label{for-graduate-students-only}
\addcontentsline{toc}{subsection}{For Graduate Students \emph{only}}

If your computer meets the
\href{http://desktop.arcgis.com/en/arcmap/10.3/get-started/system-requirements/arcgis-desktop-system-requirements.htm}{operating
system requirements} for ArcGIS and you think you'd benefit from having
access to the software at home, let Chris know (via
\href{mailto:prenercg@slu.edu}{email}).

If you are in the Public and Social Policy Ph.D.~program and your
computer meets the
\href{http://www.stata.com/support/faqs/windows/hardware-requirements/}{hardware}
and
\href{http://www.stata.com/products/compatible-operating-systems/}{software}
requirements for Stata, you should consider
\href{https://www.stata.com/order/new/edu/gradplans/student-pricing/}{purchasing
it} for yourself. I recommend purchasing a perpetual license for
Stata/IC. This is the most cost-effective solution for typical students.

\section{Buy Course Materials}\label{buy-course-materials}

\subsection*{Books}\label{books}
\addcontentsline{toc}{subsection}{Books}

There are three required books for this course:

\begin{enumerate}
\def\labelenumi{\arabic{enumi}.}
\tightlist
\item
  Brewer, Cynthia. 2015. \emph{Designing Better Maps: A Guide for GIS
  Users}. Redlands, CA: ESRI Press. ISBN-13: 978-1589484405; List Price:
  \$59.99; ebook versions available.
\item
  Gorr, Wilpen L. and Kristen S. Kurland. 2013. \emph{GIS Tutorial 1:
  Basic Workbook}. 10.3.x edition. Redlands, CA: ESRI Press. ISBN-13:
  978-1589484566; List Price: \$79.99; ebook versions available.
\item
  Thomas, Christopher and Nancy Humenik-Sappington. 2009. \emph{GIS for
  Decision Support and Public Policy Making}. Redlands, CA: ESRI Press.
  ISBN-13: 978-1589482319; List Price: \$24.95.
\end{enumerate}

There is one additional book that is optional:

\begin{itemize}
\tightlist
\item
  Mitchell, Michael N. 2010. \emph{Data Management Using Stata: A
  Practical Handbook}. College Station, TX: Stata Press. ISBN-13:
  978-1597180764; List Price: \$48.00.
\end{itemize}

Buying Mitchell (2010) is \emph{highly} recommended for graduate
students who will continue using Stata in the future and those who are
concerned about the command-line interface. I recommend waiting for a
week or two before purchasing this.

\subsection*{External Media}\label{external-media}
\addcontentsline{toc}{subsection}{External Media}

You will need a USB external storage device (either an external hard
drive or a thumb-style drive) that has at least 20GB of storage
capacity. This will be used for storing spatial data for this course.

\section{Download Course Data}\label{download-course-data}

Mots of the course data is available for download via Dropbox in a
single \texttt{.zip} file. If you want, you can let Chris know (via
\href{mailto:prenercg@slu.edu}{email}) that you'd like to download these
data before the beginning of the semester. Once you download them,
extract the data from the \texttt{.zip} file and transfer them to your
external storage device.

\chapter{Introduction to GitHub}\label{introduction-to-github}

Much of our interaction this semester outside of class will utilize
GitHub.com (or just ``GitHub''). GitHub is a web service that is a
social network for programmers, developers, data scientists,
researchers, and academics. It is also a tool for collaborating on
projects, especially projects that involve writing code.

\section{Git}\label{git}

GitHub is a web application that utilizes
\href{https://git-scm.com}{Git}:

\begin{quote}
Git is a free and open source distributed version control system
designed to handle everything from small to very large projects with
speed and efficiency.
\end{quote}

Essentially, Git is a project-wide system for tracking changes to files.
Think of it as Microsoft office's track changes feature on steroids -
every change to every file in a directory (a ``repository'' or ``repo''
in Git-lingo) is tracked. You do no need to host files online to use
Git. If you have a project saved locally (say, a doctoral thesis), you
could utilize Git to version control that project without ever uploading
it to the Internet.

For our purposes, this is just about all you need to know about Git. If
you want to learn more, \href{https://git-scm.com/about}{Git's `About'
page} is a great place to start.

\section{More Git-lingo}\label{more-git-lingo}

Beyond ``repositories'', there are a few additional terms that are
specific to Git and that are helpful to know:

\begin{itemize}
\tightlist
\item
  \textbf{Clone}: Make an identical copy of a repository on your local
  hard drive.
\item
  \textbf{Fork}:
\item
  \textbf{Commit}: Approve any changes you have made to a repository.
\item
  \textbf{Pull Request}:
\item
  \textbf{Sync}: For cloned repositories, files that have been changed
  need to be pushed to GitHub.com after they are committed.
\end{itemize}

\section{GitHub.com}\label{github.com}

GitHub is a web service that can host projects using Git's version
tracking. It is widely used by programmers, software developers, data
scientists, and academics to host and collaborate projects.

GitHub is an excellent way to backup files for a project since you can
``sync'' changes made to a repository up to GitHub's servers. It is also
an excellent way to collaborate on files with colleagues while also
using Git's version tracking. Repositories can be either public (like
all of the repos for our seminar) or private, which means that only
people who have been given access to can view the contents of the repo.
Private repos require an upgraded account, which retails for \$7/month.

Students can get access to GitHub's paid services for free, however, by
signing up for a \href{https://education.github.com}{free student
account}. This will give you access to private repositories for as long
as you are a student.

\section{GitHub Repositories}\label{github-repositories}

Users of GitHub.com adhere to a couple of norms with their repositories
that are worth knowing about. Repositories cannot have spaces in their
names (much like variables in Stata), so the naming conventions that we
will discuss in relation to Stata this semester all apply to GitHub as
well!

Public GitHub repositories also contain (typically) at least three core
files:

\begin{enumerate}
\def\labelenumi{\arabic{enumi}.}
\item
  A \textbf{license} file - since the data is out there for public
  consumption, it is important to think about how that data is licensed.
  The norm among GitHub users has been to use open source licenses,
  which let others edit and adapt your work. There are a range of
  \href{http://choosealicense.com}{licenses} that are commonly used on
  GitHub.
\item
  A \textbf{README} file - this describes the purpose and content of the
  project.
\item
  A \textbf{.gitignore} file - this stops certain types of files from
  being swept up by GitHub when a user syncs their files with a server.
\end{enumerate}

Another norm is to write using a
\href{https://en.wikipedia.org/wiki/Markup_language}{markup language}
known as \href{https://daringfireball.net/projects/markdown/}{Markdown}.
Markup languages allow users to specify exactly how they want their text
to appear when it is parsed and processed by special software. This is
different from, say, Microsoft Word, which is known as a ``what you see
is what you get'' or \textbf{WYSIWYG} editor, which uses a graphical
interface for constructing documents.

\section{Storing GitHub Repositories}\label{storing-github-repositories}

When you clone your repositories, you will be prompted to save them on
your computer. There are a number of ways in which this process can
introduce sources for trouble down the road:

\begin{enumerate}
\def\labelenumi{\arabic{enumi}.}
\tightlist
\item
  External media - storing data on devices like thumb drives or external
  hard drives can be a part of a \href{protecting-your-work.html}{backup
  workflow}. However, I have seen issues where this has appeared to
  contribute to sync errors with GitHub Desktop, particularly on
  Windows.
\item
  Cloud storage services (Dropbox, Google Drive, etc.) - like external
  drives, these services can be a part of a
  \href{protecting-your-work.html}{backup workflow}. However, like
  external drives, I have seen issues whee this has to contribute to
  sync errors with GitHub Desktop.
\end{enumerate}

In order to avoid any issues, I suggest storing GitHub repositories on
your computer's hard drive and not a thumb drive or other external
device. Make sure your are saving your files in a place not backed up to
Dropbox or another cloud storage service.

\section{GitHub Issues}\label{github-issues}

GitHub has a powerful tool for interaction called
\href{https://help.github.com/articles/about-issues/}{Issues}. These can
be accessed by opening a repository and then clicking on the ``Issues''
tab. Issues can be ``opened'' by anyone with access to the repository.
They allow for a conversation to occur in the form of messages posted
within the Issue itself. Files can be attached to Issues, and the
messages can contain Markdown formatting. Once the conversation is
complete, issues can be marked as ``closed'', which moves them into a
secondary view on the website so that they are archived.

\section{GitHub Desktop Application}\label{github-desktop-application}

\href{https://desktop.github.com}{GitHub Desktop} is a tool that allows
you to easily clone repositories hosted on GitHub, commit changes to
them, and then sync those changes up to the website. You can also create
new repositories, however this is not task you will have to do this
semester. GitHub Desktop is not a fully functional desktop version of
GitHub. For our purposes, it is important to note that the Desktop
application will not let you easily identify when repositories have been
updated by other users, view Wikis associated with repositories, or view
Issues.

\section{Learning More}\label{learning-more}

GitHub has a
\href{https://help.github.com/articles/good-resources-for-learning-git-and-github/}{resources
page} with links to websites that are great for helping you learn more
about how Git and GitHub work!

\chapter{Methods}\label{methods}

We describe our methods in this chapter.

\chapter{Applications}\label{applications}

Some \emph{significant} applications are demonstrated in this chapter.

\section{Example one}\label{example-one}

\section{Example two}\label{example-two}

\chapter{Final Words}\label{final-words}

We have finished a nice book.

\bibliography{packages.bib,book.bib}


\end{document}
